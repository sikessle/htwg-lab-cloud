\chapter{Einleitung}\textit{In diesem Kapitel werden Motivation, Zielsetzung und
Aufbau dieser Arbeit vorgestellt.}
\section{Motivation} fsdfsdfsdafasdf \cite[vgl.][S. 1]{reimann_betrieblicher_2013}.
\section{Ziel der Arbeit} Das Ziel dieser Arbeit ist es,
\section{Aufbau der Arbeit}Kurzer Einleitungssatz
Datenschutzpunkte. \begin{description}%[itemsep=-1mm]
\item [Kapitel 1] bietet einen Einstieg in das Thema der
vorliegenden Arbeit. Die Beschreibung von Motivation und Zielsetzung beleuchten die Hintergründe dieser Arbeit.
\item [Kapitel 2] befasst sich mit
\item [Kapitel 3] dient dazu, 
\item [Kapitel 4] beinhaltet 
\item [Kapitel 5] beinhaltet eine Beschreibung der
\item [Kapitel 6] definiert
\item [Kapitel 7] beschreibt 
\item [Kapitel 8] erläutert die 
\item [Kapitel 9] stellt die 
\item [Kapitel 10] resümiert im Rahmen 
\item [Kapitel 11] enthält die Schlussbetrachtung dieser Arbeit und reflektiert
in einer Zusammenfassung, welche Ziele erreicht werden konnten.
\end{description}