\chapter{OpenStack}

Kurzabriss des Kapitels

\section{Kurzbeschreibung}OpenStack ist eine Open-Source-Software der OpenStack Foundation zur Erstellung und Verwaltung privater, öffentlicher und hybrider Cloud-Umgebungen. Nach der Definition des National Institute of Standards and Technology (NIST) handelt es sich bei OpenStack um eine  Infrastructure-as-a-Service- (IaaS-) Cloud. Sowohl über RESTful APIs als auch per Weboberfläche können große Pools von virtuellen Rechen-, Speicher- und Netzwerkressourcen verwaltet werden. Durch die offene Architektur und des modularen Aufbaus bietet OpenStack eine enorm hohe Flexibilität und Skalierbarkeit. Entwickelt wird OpenStack in der Programmiersprache Python und ist unter der Apache-2.0-Lizenz lizensiert \cite[vgl.][S. 7ff]{Beitter}. 

Initiiert wurde das Projekt ursprünglich durch eine Zusammenarbeit der NASA und von Rackspace. Bedingt dadurch, dass Wissenschaftler nach Beendingung ihrer Experimente die dafür eigens angeschaffte Hardware nicht mehr benötigten, bestand das Ziel von OpenStack darin, Rechenkapazität an zentraler Stelle zu bündeln und diese auf Abruf bereitzustellen \cite[vgl.][]{Loschwitz}. Zwischenzeitlich erstreckt sich die Entwickler- und Nutzergemeinde über 130 Länder und wird von einer Vielzahl an namhaften Unternehmen, unter anderem Hewlett-Packard, IBM, RedHat, Ubuntu, SUSE Linux GmbH, Dell, PayPal, Cisco Systems und Yahoo, unterstützt \cite[vgl.][S. 78]{Srinivasan2014}.

Das Projekt, das seit dem ersten Release im Jahr 2010 eine stetige Fortentwickung verfolgt, liegt inzwischen in der Version 2015.1 vor. Mit jedem neuem Release sind weitere Komponenten hinzugekommen, die eigenständige Projekte darstellen und verschiedene Aufgaben erfüllen. Die wichtigsten dieser Komponenten sollen nachfolgend kurz erläutert werden.

\section{Komponenten}
