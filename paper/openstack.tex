\chapter{OpenStack}

Kurzabriss des Kapitels

\section{Kurzbeschreibung}OpenStack ist eine Open-Source-Software zur Erstellung und Verwaltung privater, öffentlicher und hybrider Cloud-Umgebungen. Nach der Definition des National Institute of Standards and Technology (NIST) handelt es sich bei OpenStack um eine Infrastructure-as-a-Service- (IaaS-) Cloud. Sowohl über RESTful APIs als auch per Weboberfläche können große Pools von virtuellen Rechen-, Speicher- und Netzwerkressourcen verwaltet werden. Durch die offene Architektur und den modularen Aufbau bietet OpenStack eine enorm hohe Flexibilität und Skalierbarkeit. Entwickelt wird OpenStack in der Programmiersprache Python und ist unter der Apache-2.0-Lizenz lizensiert \cite[vgl.][S. 7ff]{Beitter}. 

Initiiert wurde das Projekt ursprünglich durch eine Zusammenarbeit der NASA und von Rackspace. Bedingt dadurch, dass Wissenschaftler nach Beendingung ihrer Experimente die dafür eigens angeschaffte Hardware nicht mehr benötigten, bestand das Ziel von OpenStack darin, Rechenkapazität an zentraler Stelle zu bündeln und diese auf Abruf bereitzustellen \cite[vgl.][]{Loschwitz}. Mittlerweile wird die Entwicklung und Verteilung von OpenStack von einer Organisation, der OpenStack Foundation, koordiniert, die derzeit mehr als 18.000 Mitglieder aus 140 Ländern auf der ganzen Welt besitzt. Wichtige strategische und finanzielle Entscheidungen werden von einem Aufsichtsrat getroffen. Zu den Mitgliedern zählen unter anderem AT&T, Hewlett-Packard, IBM, RedHat, SUSE Linux GmbH und Cisco Systems \cite[vgl.][]{OpenStackFoundation}.

\section{Komponenten}Das Projekt, das seit dem ersten Release im Jahr 2010 eine stetige Fortentwickung verfolgt, liegt inzwischen in der Version 2015.1 vor. Mit jedem neuem Release sind weitere Komponenten hinzugekommen, die eigenständige Projekte darstellen und verschiedene Aufgaben erfüllen. Die wichtigsten dieser Komponenten sollen nachfolgend kurz erläutert werden.
\begin{description}
\item[Keystone]
\item[Glance]
\item[Nova]
\item[Cinder]
\item[Swift]
\item[Neutron]
\item[Horizon]
\item[Ceilometer]
\item[Heat]
\end{description}
