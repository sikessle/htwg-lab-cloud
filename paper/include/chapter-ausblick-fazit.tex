\chapter{Ausblick und Fazit}

Die entwickelte HTWG Lab Cloud stellt die nötige Infrastruktur zur Verfügung, um die Laborstunden effizienter zu nutzen.
Insbesondere Studenten können davon profitieren, dass sie die kurze Zeit für die wichtigen und relevanten Übungsaufgaben verwenden können und nicht mit der Installation und Konfiguration von Programmen beschäftigt sind.
Durch die Single-Sign-On Lösung mit LDAP und die Anbindung an Moodle wurde eine komfortable Umgebung geschaffen, die es den Professoren leicht ermöglicht eine Übungsstunde vorzubereiten und zu starten.
Das angepasste Dashboard bietet mit der Web-Oberfläche eine gut verständliche Schnittstelle für die Professoren. 
Somit richtet sich das Projekt auch ausdrücklich an nicht-technische Menschen.
Momentan liegt aus lizenzrechtlichen Gründen als Basis-Image lediglich Ubuntu vor. 
Für sehr interaktive Anwendungen wie Videoschnitt o.ä. ist die momentane Leistung der VNC Lösung nicht ausgelegt. 
Die aufgeführten Probleme während der Entwicklung kosteten viel Zeit und hätten gerne in weitere Funktionen der HTWG Lab Cloud investiert werden können.
Jedoch konnten wir durch die intensive Beschäftigung mit dem Projekt viele neue und bekannte Technologien lernen bzw. vertiefen, wie z. B. OpenStack, Python, Shell-Skripte, HTML, CSS, Linux, LDAP und Web-Services.
Auch die Teamarbeit und vor allem Kollaboration über Git bzw. GitHub wurde geübt. 
Regelmäßige Meetings und der reale Nutzen des Projekts haben uns auch in stressigen Phasen weiter motiviert.
Schlussendlich bleibt das gute Gefühl, dass hier ein Projekt geschaffen wurde, dass vielleicht in einiger Zeit tatsächlich bei der HTWG oder anderen Hochschulen zum Einsatz kommen könnte.

Wir laden auch noch weitere Studenten ein, dieses Projekt weiter voranzutreiben. 
So könnte das Dashboard noch um weitere Managementfunktionen erweitert und das Design angepasst werden.
Auch ein Blob-Storage mit Swift für große Objekte (Binaries, etc.) könnte implementiert werden.
So könnte automatisch pro Moodle-Kurs ein Blob-Storage erstellt werden, auf den die Studenten des jeweiligen Kurses automatisch Lesezugriff besitzen.
Hier wäre die Entwicklung einer graphischen Oberfläche für den Professor und die Anbindung an die Instanzen ein mögliches Projektthema.
Weitere Basis-Images mit z. B. Windows würden den Nutzen der HTWG Lab Cloud weiter erhöhen.
Zuletzt sollte auch die konkrete Installation beim Rechenzentrum angestrebt werden.
Dieses müsste bei der Implementierung unterstützt und die Hardwareanforderungen für die HTWG analysiert werden.

