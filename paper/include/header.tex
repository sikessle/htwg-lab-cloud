%
%
% Latex Einstellungen
%
%

\documentclass[a4paper,
				11pt,
				headsepline,				% Linie nach Kopfzeile
				headinclude,
				numbers=noenddot,
				bibliography=totoc,	% Aufführen des Literaturverzeichnisses
				listof=totoc,
				parskip=half,				% Zwischen Absätzen entsteht eine halber Abstand
				cleardoublepage=empty,
				ngerman,				
				oneside,						% Einseitig (für den nicht-Druck)
				openany					% unterdrückt leere Seiten vor Kapiteln
				]{scrreprt}

% verbesserte Font-Darstellung
\usepackage{ae}
%Tabellenunterschriften verkleinern
\usepackage[font=small]{caption}
%Häkchen
\usepackage{amssymb}
\usepackage{amsmath}
% Zeichenkodierung
\usepackage[utf8]{inputenc}
\usepackage{textcomp}
% Positionierung von Figuren
\usepackage{wrapfig} 
% Verwendung von Microtype für Zeilenumbrüche etc.
\usepackage[activate={true,nocompatibility},
				final,
				tracking=true,
				kerning=true,
				spacing=true,
				factor=1100,
				stretch=10,
				shrink=10]{microtype}
% neue deutsche Rechtschreibung
\usepackage[ngerman]{babel}
% Zeilenabstand
\usepackage[onehalfspacing]{setspace}
% Anführungszeichen 
\usepackage[babel,german]{csquotes}
% Schriften
\usepackage[scaled=.90]{helvet}
\setkomafont{descriptionlabel}{\normalfont\bfseries}
% kleinerer abstand bei listen
\usepackage{mdwlist}
% Farben
\usepackage{color}
\usepackage{colortbl}
\definecolor{lightgrey}{gray}{0.95}
\definecolor{dunkelgrau}{rgb}{0.8,0.8,0.8}
\definecolor{LinkColor}{rgb}{0,0,0.2}
% für Tabellenlinien
\usepackage{booktabs}
\usepackage{hhline}
\usepackage{tabularx}
\usepackage{tabulary}
% für lange Tabellen
\usepackage{longtable}
\usepackage{filecontents}
\usepackage{ltxtable}
% Abstand zwischen Spalten
\setlength{\tabcolsep}{10pt} 
% Zeilenabstand
\renewcommand{\arraystretch}{1.5} 
% Gliederungstiefe im Inhaltsverzeichnis ändern
\setcounter{tocdepth}{3}
\usepackage{multirow}
\usepackage{bigstrut}
% Grafiken
\usepackage{graphicx}
\usepackage{placeins}

% Silbentrennung
\clubpenalty = 10000
\widowpenalty = 10000
\displaywidowpenalty = 10000
\usepackage{pdfpages}
% PDF Optionen
\usepackage[bookmarks=true,
            bookmarksopen=true,
            bookmarksnumbered=true,
            bookmarksopenlevel=2,%
            hyperindex=true,
            hyperfootnotes=false,
            plainpages=false,
            pdfstartview=FitV,
            pdfpagelabels=true,
            linkcolor=black,
            colorlinks=true,
            citecolor=black,
            urlcolor=black,
            pdfauthor={Dennis Parlak, Simon Kessler, Tobias Keh, Marco Grupe},
            pdftitle={HTWG Lab Cloud mit OpenStack},
            pdfsubject={Ausarbeitung},
            pdfdisplaydoctitle=true, % Dokumenttitel statt Dateiname anzeigen.
            pdflang=de, % Sprache des Dokuments.
            pdfkeywords={OpenStack, HTWG}, ]{hyperref}                 

%Auflistungsabstände kleiner machen
\usepackage{paralist}
% Abstand zweier Listenelemente kleiner
\setlength{\itemsep}{0ex plus0.2ex}
\renewcommand{\floatpagefraction}{.8}
\renewcommand{\textfraction}{.10}
\setlength{\footnotesep}{4mm}
\usepackage{float}
% Überschriften nach irgendwelchen DIN-Normen
\renewcommand{\bibname}{Quellenverzeichnis}    
\renewcommand{\contentsname}{Inhalt}    
% Kopfzeilen enthalten Üerschriften oberster Ebene
\pagestyle{headings}
% fullref für Kapitel
\newcommand{\fullref}[1]{Kapitel~\ref{#1} \textit{(\nameref{#1})} auf Seite \pageref{#1}}
\newcommand{\fullrefapdx}[1]{Anhang~\ref{#1} \textit{(\nameref{#1})} auf Seite \pageref{#1}}
% fullref fuer Abschnitte
\newcommand{\fullrefsct}[1]{Abschnitt \textit{\nameref{#1}} auf Seite \pageref{#1}}
\usepackage{enumitem}
\usepackage{epigraph}
\usepackage{quotchap}
\usepackage{bookmark}	
% Quellcode Listings
\usepackage{listings}
% Listings formatieren
\lstset{%
basicstyle=\ttfamily,
breaklines=true,
keepspaces
}
% Umlaute in Listings erlauben
\lstset{literate=%
{Ö}{{\"O}}1
{Ä}{{\"A}}1
{Ü}{{\"U}}1
{ß}{{\ss}}2
{ü}{{\"u}}1
{ä}{{\"a}}1
{ö}{{\"o}}1
}
% \code{...} für inline Code
\newcommand{\code}[1]{\lstinline!#1!}
% Literaturverzeichnis
\usepackage[backend=biber, hyperref=true]{biblatex}
\bookmarksetup{startatroot}
\addbibresource{include/latexlit.bib}


% Titel, Autoren, Betreuer, Studiengang
\newcommand{\thema}{HTWG Lab Cloud mit OpenStack}
\newcommand{\betreuer}{Prof. Dr. Hanno Langweg}
\newcommand{\autor}{Dennis Parlak, Simon Kessler, Tobias Keh, Marco Grupe}
\newcommand{\studiengang}{MSI Informatik}
% Abstract
\newcommand{\zusammenfassung}{%
Im Rahmen der Vorlesung \enquote{Cloud Application Development} bei Herrn \betreuer{}  wurde die Cloud-basierte OpenStack Infrastruktur \emph{HTWG Lab Cloud} zur Unterstützung der HTWG-Labore entworfen und entwickelt. 
Dies beinhaltet die Konzeption des Systems, den Workflow einer Übungsstunde und die Entwicklung mit DevStack, einer OpenStack Entwicklungsumgebung. 
Die HTWG Lab Cloud bietet die Möglichkeit, interaktive Übungsstunden in den Laboren effektiver zu nutzen.
Durch die Bereitstellung von angepassten virtuellen Rechnern, wird keine Zeit für die oftmals unnötig komplizierte Installation und Konfiguration von Tools verschwendet.
Durch Virtualisierung der einzelnen Laborrechner, können mit einem Klick alle Rechner mit einem vom Professor angepassten Betriebssystem-Image hochgefahren werden.
Durch die enge Anbindung an Moodle, kann so ein kompletter Moodle-Kurs mit entsprechenden Laborrechnern versorgt werden.
Die Studenten erhalten daraufhin eine E-Mail mit der URL, um sich per VNC mit den Rechnern zu verbinden. 
Über LDAP wird eine Single-Sign-On Lösung angeboten, sodass die gewohnten HTWG Zugangsdaten für die Rechner als auch die Administrator-Oberfläche gelten.
Des Weiteren stehen im entwickelten Betriebssystem-Image das HTWG home-Laufwerk und ein \enquote{lab-drive} zur dauerhaften Speicherung von Daten zur Verfügung.
Ein angepasstes Web-Dashboard bietet den Professoren einen einfachen und unkomplizierten Zugang zur HTWG Lab Cloud.
}

