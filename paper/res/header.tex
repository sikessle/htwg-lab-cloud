%
%
% Latex Einstellungen
%
%
\documentclass[a4paper,			% Papiersorte
				BCOR10mm,
				11pt,				% Schriftgroesse
				headsepline,		% Linie nach Kopfzeile
				headinclude,
				numbers=noenddot,
				%footsepline,		% Linie vor Fusszeile
				bibliography=totoc,	% Aufführen des Literaturverzeichnisses
				listof=totoc,
				parskip=half,		% Zwischen Absätzen entsteht eine halbe Seite Abstand
				cleardoublepage=empty,
				ngerman,			% Übersetzer
				oneside,				% Einseitig
				openany			% unterdrückt leere Seiten vor Kapiteln
				]{scrreprt}
\let\cleardoublepage\clearpage
% verbesserte Font-Darstellung
\usepackage{ae}
%Tabellenunterschriften verkleinern
\usepackage[font=small]{caption}
%Häkchen
\usepackage{amssymb}
% Zeichenkodierung
\usepackage[utf8]{inputenc}
\usepackage{textcomp}
\usepackage{wrapfig} % Paket zur Positionierung einbinden
% Verwendung von Microtype für Zeilenumbrüche etc.
\usepackage[activate={true,nocompatibility},final,tracking=true,kerning=true,spacing=true,factor=1100,stretch=10,shrink=10]{microtype}
% neue deutsche Rechtschreibung
\usepackage[ngerman]{babel}
\usepackage{amsmath}

% Zeilenabstand
\usepackage[onehalfspacing]{setspace}
% Anführungszeichen 
\usepackage[babel,german]{csquotes}

%Schriften
\usepackage[scaled=.90]{helvet}
\setkomafont{descriptionlabel}{\normalfont\bfseries}
%kleinerer abstand bei listen
\usepackage{mdwlist}
% Farben
\usepackage{color}
\usepackage{colortbl}
\definecolor{lightgrey}{gray}{0.95}
\definecolor{dunkelgrau}{rgb}{0.8,0.8,0.8}
\definecolor{LinkColor}{rgb}{0,0,0.2}
% Literaturverweise mit (Autor Jahr) nach DIN
\usepackage[backend=biber]{biblatex}
%\renewcommand{\cite}{\citep} % citep zu cite umgebogen.

% für Tabellenlinien
\usepackage{booktabs}
\usepackage{hhline}
\usepackage{tabularx}
\usepackage{tabulary}
% für lange Tabellen
\usepackage{longtable}
\usepackage{filecontents}
\usepackage{ltxtable}
\setlength{\tabcolsep}{10pt} % Abstand zwischen Spalten
\renewcommand{\arraystretch}{1.5} % Zeilenabstand
% Gliederungstiefe im Inhaltsverzeichnis ändern
\setcounter{tocdepth}{3}
\usepackage{multirow}
\usepackage{bigstrut}
% Grafiken
\usepackage{graphicx}

% Silbentrennung
\clubpenalty = 10000
\widowpenalty = 10000
\displaywidowpenalty = 10000
\usepackage{pdfpages}
% PDF Optionen
\usepackage[bookmarks=true,
            bookmarksopen=true,
            bookmarksnumbered=true,
            bookmarksopenlevel=2,%
            hyperindex=true,
            hyperfootnotes=false,
            plainpages=false,
            pdfstartview=FitV,
            pdfpagelabels=true,
            linkcolor=black,
            colorlinks=true,
            citecolor=black,
            urlcolor=black,
            pdfauthor={Dennis Parlak, Simon Kessler, Tobias Keh, Marco Grupe},
            pdftitle={HTWG Lab Cloud mit OpenStack},
            pdfsubject={Ausarbeitung},
            pdfdisplaydoctitle=true, % Dokumenttitel statt Dateiname anzeigen.
            pdflang=de, % Sprache des Dokuments.
            pdfkeywords={OpenStack, HTWG}, ]{hyperref}                 

%\hypersetup{%
%	colorlinks=false, % Aktivieren von farbigen Links im Dokument
%	linkcolor=LinkColor, % Farbe festlegen
%	citecolor=LinkColor,
%	filecolor=LinkColor,
%	menucolor=LinkColor,
%	urlcolor=LinkColor,
%	bookmarksnumbered=true % Überschriftsnummerierung im PDF Inhalt anzeigen.
%}

%Auflistungsabstände kleiner machen
\usepackage{paralist}
% für Glossar
\usepackage[intoc,german]{nomencl}
\let\abbrev\nomenclature
\renewcommand{\nomname}{Abkürzungsverzeichnis}
\setlength{\nomlabelwidth}{.19\hsize}
\renewcommand{\nomlabel}[1]{#1 \dotfill}
\setlength{\nomitemsep}{-\parsep}
\makenomenclature

\usepackage[normalem]{ulem}
\newcommand{\markup}[1]{\uline{#1}}

% Buchumschlag Rändereinstellung
\usepackage{vmargin}

% Seitengröߟe
\setpapersize{A4}

% eine Randnote fuer fehlende Teile 
\newcommand{\fehlt}[1]{(\marginpar[\hfill!$\longrightarrow$]{$\longleftarrow$!}Hier fehlt \emph{#1})}

% Absatzabstand etwas groesser
\setlength{\parskip}{1.5ex plus0.5ex minus0.5ex}

% Abstand zweier Listenelemente kleiner
\setlength{\itemsep}{0ex plus0.2ex}

\renewcommand{\floatpagefraction}{.8}
\renewcommand{\textfraction}{.10}
\setlength{\footnotesep}{4mm}
\usepackage{float}
% Überschriften nach irgendwelchen DIN-Normen
\renewcommand{\bibname}{Quellenverzeichnis}    
\renewcommand{\contentsname}{Inhalt}    
    
% Kopfzeilen enthalten Üœberschriften oberster Ebene
\pagestyle{headings}

%fullref für Kapitel
\newcommand{\fullref}[1]{Kapitel~\ref{#1} \textit{(\nameref{#1})} auf Seite \pageref{#1}}
\newcommand{\fullrefapdx}[1]{Anhang~\ref{#1} \textit{(\nameref{#1})} auf Seite \pageref{#1}}

%fullref fuer Abschnitte
\newcommand{\fullrefsct}[1]{Abschnitt \textit{\nameref{#1}} auf Seite \pageref{#1}}

% Fußoten seitenweise nummerieren
\usepackage{footnpag}
\usepackage{enumitem}
\usepackage{epigraph}
\usepackage{quotchap}
\usepackage{bookmark}	

\usepackage{listings}

% Listings formatieren
\lstset{%
basicstyle=\ttfamily,
breaklines=true,
keepspaces
}

\lstset{literate=%
{Ö}{{\"O}}1
{Ä}{{\"A}}1
{Ü}{{\"U}}1
{ß}{{\ss}}2
{ü}{{\"u}}1
{ä}{{\"a}}1
{ö}{{\"o}}1
}

\newcommand{\code}[1]{\lstinline!#1!}

\addbibresource{res/latexlit.bib}