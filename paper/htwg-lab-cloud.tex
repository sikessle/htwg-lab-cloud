% !TEX encoding = UTF-8 Unicode
%===========================================================
% MSI - Cloud Application Development - Projekt 
% Autoren: Dennis Parlak, Simon Kessler, Tobias Keh, Marco Grupe
% Hochschule für Technik, Wirtschaft und Gestaltung, Konstanz
% Fakultät Informatik
%===========================================================

%
%
% Latex Einstellungen
%
%

\documentclass[a4paper,
				11pt,
				headsepline,				% Linie nach Kopfzeile
				headinclude,
				numbers=noenddot,
				bibliography=totoc,	% Aufführen des Literaturverzeichnisses
				listof=totoc,
				parskip=half,				% Zwischen Absätzen entsteht eine halber Abstand
				cleardoublepage=empty,
				ngerman,				
				oneside,						% Einseitig (für den nicht-Druck)
				openany					% unterdrückt leere Seiten vor Kapiteln
				]{scrreprt}

% verbesserte Font-Darstellung
\usepackage{ae}
%Tabellenunterschriften verkleinern
\usepackage[font=small]{caption}
%Häkchen
\usepackage{amssymb}
\usepackage{amsmath}
% Zeichenkodierung
\usepackage[utf8]{inputenc}
\usepackage{textcomp}
% Positionierung von Figuren
\usepackage{wrapfig} 
% Verwendung von Microtype für Zeilenumbrüche etc.
\usepackage[activate={true,nocompatibility},
				final,
				tracking=true,
				kerning=true,
				spacing=true,
				factor=1100,
				stretch=10,
				shrink=10]{microtype}
% neue deutsche Rechtschreibung
\usepackage[ngerman]{babel}
% Zeilenabstand
\usepackage[onehalfspacing]{setspace}
% Anführungszeichen 
\usepackage[babel,german]{csquotes}
% Schriften
\usepackage[scaled=.90]{helvet}
\setkomafont{descriptionlabel}{\normalfont\bfseries}
% kleinerer abstand bei listen
\usepackage{mdwlist}
% Farben
\usepackage{color}
\usepackage{colortbl}
\definecolor{lightgrey}{gray}{0.95}
\definecolor{dunkelgrau}{rgb}{0.8,0.8,0.8}
\definecolor{LinkColor}{rgb}{0,0,0.2}
% für Tabellenlinien
\usepackage{booktabs}
\usepackage{hhline}
\usepackage{tabularx}
\usepackage{tabulary}
% für lange Tabellen
\usepackage{longtable}
\usepackage{filecontents}
\usepackage{ltxtable}
% Abstand zwischen Spalten
\setlength{\tabcolsep}{10pt} 
% Zeilenabstand
\renewcommand{\arraystretch}{1.5} 
% Gliederungstiefe im Inhaltsverzeichnis ändern
\setcounter{tocdepth}{3}
\usepackage{multirow}
\usepackage{bigstrut}
% Grafiken
\usepackage{graphicx}
\usepackage{placeins}

% Silbentrennung
\clubpenalty = 10000
\widowpenalty = 10000
\displaywidowpenalty = 10000
\usepackage{pdfpages}
% PDF Optionen
\usepackage[bookmarks=true,
            bookmarksopen=true,
            bookmarksnumbered=true,
            bookmarksopenlevel=2,%
            hyperindex=true,
            hyperfootnotes=false,
            plainpages=false,
            pdfstartview=FitV,
            pdfpagelabels=true,
            linkcolor=black,
            colorlinks=true,
            citecolor=black,
            urlcolor=black,
            pdfauthor={Dennis Parlak, Simon Kessler, Tobias Keh, Marco Grupe},
            pdftitle={HTWG Lab Cloud mit OpenStack},
            pdfsubject={Ausarbeitung},
            pdfdisplaydoctitle=true, % Dokumenttitel statt Dateiname anzeigen.
            pdflang=de, % Sprache des Dokuments.
            pdfkeywords={OpenStack, HTWG}, ]{hyperref}                 

%Auflistungsabstände kleiner machen
\usepackage{paralist}
% Abstand zweier Listenelemente kleiner
\setlength{\itemsep}{0ex plus0.2ex}
\renewcommand{\floatpagefraction}{.8}
\renewcommand{\textfraction}{.10}
\setlength{\footnotesep}{4mm}
\usepackage{float}
% Überschriften nach irgendwelchen DIN-Normen
\renewcommand{\bibname}{Quellenverzeichnis}    
\renewcommand{\contentsname}{Inhalt}    
% Kopfzeilen enthalten Üerschriften oberster Ebene
\pagestyle{headings}
% fullref für Kapitel
\newcommand{\fullref}[1]{Kapitel~\ref{#1} \textit{(\nameref{#1})} auf Seite \pageref{#1}}
\newcommand{\fullrefapdx}[1]{Anhang~\ref{#1} \textit{(\nameref{#1})} auf Seite \pageref{#1}}
% fullref fuer Abschnitte
\newcommand{\fullrefsct}[1]{Abschnitt \textit{\nameref{#1}} auf Seite \pageref{#1}}
\usepackage{enumitem}
\usepackage{epigraph}
\usepackage{quotchap}
\usepackage{bookmark}	
% Quellcode Listings
\usepackage{listings}
% Listings formatieren
\lstset{%
basicstyle=\ttfamily,
breaklines=true,
keepspaces
}
% Umlaute in Listings erlauben
\lstset{literate=%
{Ö}{{\"O}}1
{Ä}{{\"A}}1
{Ü}{{\"U}}1
{ß}{{\ss}}2
{ü}{{\"u}}1
{ä}{{\"a}}1
{ö}{{\"o}}1
}
% \code{...} für inline Code
\newcommand{\code}[1]{\lstinline!#1!}
% Literaturverzeichnis
\usepackage[backend=biber, hyperref=true]{biblatex}
\bookmarksetup{startatroot}
\addbibresource{include/latexlit.bib}


% Titel, Autoren, Betreuer, Studiengang
\newcommand{\thema}{HTWG Lab Cloud mit OpenStack}
\newcommand{\betreuer}{Prof. Dr. Hanno Langweg}
\newcommand{\autor}{Dennis Parlak, Simon Kessler, Tobias Keh, Marco Grupe}
\newcommand{\studiengang}{MSI Informatik}
% Abstract
\newcommand{\zusammenfassung}{%
Im Rahmen der Vorlesung \enquote{Cloud Application Development} bei Herrn \betreuer{}  wurde die Cloud-basierte OpenStack Infrastruktur \emph{HTWG Lab Cloud} zur Unterstützung der HTWG-Labore entworfen und entwickelt. 
Dies beinhaltet die Konzeption des Systems, den Workflow einer Übungsstunde und die Entwicklung mit DevStack, einer OpenStack Entwicklungsumgebung. 
Die HTWG Lab Cloud bietet die Möglichkeit, interaktive Übungsstunden in den Laboren effektiver zu nutzen.
Durch die Bereitstellung von angepassten virtuellen Rechnern, wird keine Zeit für die oftmals unnötig komplizierte Installation und Konfiguration von Tools verschwendet.
Durch Virtualisierung der einzelnen Laborrechner, können mit einem Klick alle Rechner mit einem vom Professor angepassten Betriebssystem-Image hochgefahren werden.
Durch die enge Anbindung an Moodle, kann so ein kompletter Moodle-Kurs mit entsprechenden Laborrechnern versorgt werden.
Die Studenten erhalten daraufhin eine E-Mail mit der URL, um sich per VNC mit den Rechnern zu verbinden. 
Über LDAP wird eine Single-Sign-On Lösung angeboten, sodass die gewohnten HTWG Zugangsdaten für die Rechner als auch die Administrator-Oberfläche gelten.
Des Weiteren stehen im entwickelten Betriebssystem-Image das HTWG home-Laufwerk und ein \enquote{lab-drive} zur dauerhaften Speicherung von Daten zur Verfügung.
Ein angepasstes Web-Dashboard bietet den Professoren einen einfachen und unkomplizierten Zugang zur HTWG Lab Cloud.
}



\newcommand{\thema}{HTWG Lab Cloud mit OpenStack}
\newcommand{\zusammenfassung}{Hier kommt eine kurze Zusammenfassung der Arbeit}

\newcommand{\ausgabedatum}{}
\newcommand{\abgabedatum}{}
\newcommand{\autor}{Dennis Parlak, Simon Kessler, Tobias Keh, Marco Grupe}
\newcommand{\autorStrasse}{}
\newcommand{\autorPLZ}{}
\newcommand{\autorOrt}{}
\newcommand{\autorGeburtsort}{}
\newcommand{\autorGeburtsdatum}{}
\newcommand{\prueferA}{Prof. Dr. Hanno Langweg}
\newcommand{\prueferB}{}
\newcommand{\firma}{}
\newcommand{\studiengang}{MSI Informatik}
\bibliographystyle{alphadin}
\raggedbottom 
\begin{document}
\frontmatter
\begin{titlepage}

\vspace*{-3.5cm}

\begin{flushleft}
\hspace*{-1cm} \includegraphics[width=15.7cm]{htwg-logo}
\end{flushleft}

\vspace{2.5cm}

\begin{center}
	\huge{
		\textbf{HTWG Lab Cloud mit OpenStack} \\[5cm]
	}
	\Large{
		\textbf{Dennis Parlak, Simon Kessler,\\Tobias Keh, Marco Grupe}} \\[6.5cm]
	\large{
		\textbf{Konstanz, 01.07.2015} \\[2.3cm]
	}
	
	\Huge{
		\textbf{{\sf PROJEKTARBEIT}}
	}
\end{center}

\end{titlepage}
\begin{center}
{\Large \textbf{Zusammenfassung (Abstract)}}
\end{center}

\bigskip

\begin{center}
	\begin{tabular}{p{2.8cm}p{10cm}}
		Thema: & \thema \\
		 & \\
		Teilnehmer: & \autor \\
		 & \\
		Betreuer: & \betreuer \\
		 & \\
	\end{tabular}
\end{center}

\bigskip

\noindent
\zusammenfassung
% Inhaltsverzeichnis
\setcounter{tocdepth}{1}
\tableofcontents 
% Abbildungsverzeichnis
\listoffigures
% Tabellenverzeichnis
\listoftables
% Abkürzungsverzeichnis
%\printnomenclature
%\include{res/abbreviations}
\mainmatter
\pagenumbering{arabic}
		
% ------------------------------------------ CHAPTER
\chapter{OpenStack}

Kurzabriss des Kapitels

\section{Kurzbeschreibung}

\section{Komponenten}

% ------------------------------------------ CHAPTER
\chapter{HTWG Lab Cloud}

Kurzabriss des Kapitels

Source-Code unter \url{http://github.com/sikessle/htwg-lab-cloud}

\section{Zielsetzung}

\section{Use Cases}

\section{Workflow und Organisation}

\section{Dashboard}

\section{OpenStack LDAP-Authentifizierung}

\section{Anbindung an Moodle}

% ------------------------------------------ Simon
\section{Lab-Compute Image}

Da die OpenStack Umgebung gewisse Anforderungen an das Instanz-Image stellt, wurde in diesem Projekt ein Basis-Image entwickelt. 
Dieses kann in der OpenStack Umgebung gestartet und vom Professor für die jeweilige Übung angepasst werden.
Grundsätzlich sind in OpenStack Windows XP, 7, 8 und Linux-basierte Betriebssysteme als Instanz konfigurierbar. 
Aus lizenzrechtlichen Gründen wurde die Entscheidung für das Linux-basierte Betriebssystem \emph{Ubuntu} getroffen. 
Prinzipiell lassen sich die hier durchgeführten Anpassungen auch auf Windows übertragen, wobei die konkrete Konfiguration abweicht.

Der Workflow, um ein Image anzupassen und nachfolgend Instanzen mit diesem den Studenten bereitzustellen ergibt sich wie folgt:

\begin{itemize}
\item Starten des Basis-Images in einer Instanz durch den Professor.
\item Installation von Programmen, Konfiguration der Instanz und andere Anpassungen.
\item Erstellen eines Snapshots, um die Änderungen festzuhalten. Dies erzeugt ein neues Image.
\item Starten der einzelnen Instanzen für die Studenten. Dabei wird die Instanz automatisch an den Student angepasst, sodass nur dieser Zugriff mittels seiner im LDAP hinterlegten Zugangsdaten hat.
\end{itemize}

Im Folgenden werden auf die einzelnen Komponenten, die im Zuge dieser Projektarbeit erarbeitet wurden, eingegangen.

\subsection{Details}

Als Betriebssystem wurde Ubuntu 14.04 Desktop LTS ausgewählt. 
Dieses ist zwar nicht die neuste Ubuntu-Version, bietet jedoch aufgrund des \emph{Long Term Support} Updates und relevante Sicherheitsfixes für 5 Jahre. 
Somit ist gewährleistet, dass das Basis-Image 2019 mit relevanten Updates versorgt wird. 
Insbesondere unter Betrachtung des Support-Aufwands durch das HTWG-Rechenzentrums, soll so eine lange Laufzeit gewährleistet werden.

Aufgrund der Dateigröße ($> 5$ GB) kann das Image nicht in Git eingecheckt werden und wird über Dropbox unter \url{https://www.dropbox.com/sh/0dgcaxc2u7cbfgi/AAAgnS87aYXRp6HJzK7oWSrma?dl=0} bereitgestellt. Das Image kann auch direkt bei den Autoren der Projektarbeit angefordert werden.

Da das Basis-Image noch keine personenbezogenen Daten enthält, wird ein Standard-Account mit sudo Rechten bereitgestellt:

\begin{itemize}
\item \emph{Benutzer:} ubuntu (Vollständiger Name: HTWG Lab Cloud)
\item \emph{Passwort:} ubuntu
\end{itemize}

Das von uns bereitgestellte Image enthält bereits einige Standardprogramme:

\begin{itemize}
\item Unity als Desktop-Manager
\item Dropbox
\item git
\item openssh-server
\item vim
\item OpenOffice
\item Firefox
\item und weitere Office-Applikationen
\end{itemize}

Auf andere Cloudspeicher-Anbieter musste verzichtet werden, da Google Drive und Windows OneDrive Linux nicht nativ unterstützen.

\subsection{Cloud-Kompatibilität}

OpenStack stellt gewisse Anforderungen an die Gast-Betriebssysteme, die sich im OpenStack Handbuch \cite{osImageRequirement} nachschlagen lassen. 
Die initiale Konfiguration des Basis-Images wurde aus Performance-Gründen in VirtualBox durchgeführt.
Durch Installation einiger Packete und Anpassung von Konfigurationsdateien, wurde das Ubuntu Image OpenStack kompatibel gemacht:

\begin{itemize}
\item Installation von cloud-init, cloud-utils und cloud-initramfs-growroot *
\item Anpassen der Kernel-Boot-Parameter in der Datei /etc/default/grub. Dort wurde der Parameter \emph{console=tty,ttyS0} ergänz. Ein anschließendes \emph{update-grub} aktualisert den Grub-Bootloader. Diese Änderung aktiviert die Ausgabe der Log-Meldungen im Horizon-Dashboard.
\item Anpassen von \emph{cloud.cfg} des Pakets cloud-init. Dieses deaktiviert standardmäßig den Passwort-Login des Images. Da Nutzer jedoch per LDAP bzw. mit dem Standardbenutzername sich anmelden können sollen, muss dies aktiviert werden. *
\end{itemize}

Die mit * gekennzeichneten Punkte werden von dem entworfenen Skript \emph{cloud-init-script.sh} automatisch ausgeführt. 
Anschließend löscht sich das Skript und fährt das Betriebssystem herunter.
Da nach Ausführung des \emph{cloud-init-script.sh} Skripts das Image in VirtualBox nicht mehr ohne weiteres gestartet werden kann, sollte das Skript am Ende der Konfiguration ausgeführt werden.

\subsection{Konvertierung von VirtualBox nach OpenStack}

Nachdem das Basis-Image konfiguriert und kompatibel zu OpenStack gemacht wurde muss das Image noch in ein Format gewandelt werden, das OpenStack versteht.
Das entwickelte Skript \emph{image-virtualbox-to-openstack.sh} führt diese Umwandlung aus. 
Die Benutzung erfolgt mit folgendem Aufruf: \emph{./image-virtualbox-to-openstack.sh <Pfad zum gewünschten VirtualBox Image Snapshot>}.
Im Wesentlichen wird dabei mittels \emph{VBoxManage} das Image in ein \emph{raw} Format gewandelt und anschließend mit \emph{qemu-img} nach \emph{qcow2} überführt.

\subsection{LDAP-Authentifizierung}

Screenshot Graphischer Login

Read-Only
Graphischer Login
Nur ein einzelner User pro VM

Nötige Attribute:
- uid
- uidNumber
- gidNumber
- homeDirectory
- loginShell

HTWG LDAP
URL: ldap.htwg-konstanz.de
BaseDN: dc=fh-konstanz,dc=de
Nutzer: ou=users,dc=fh-konstanz,dc=de

Attribut-Mapping

Headless Installation

\subsection{HTWG home-Laufwerk}

smb://homedrive.htwg-konstanz.de/home

Auth mit LDAP
PAM

\subsection{Block-Storage}

Zweck
Cinder
fstab
udev

\subsection{Snapshots}

Basis-Image -> Snapshot -> Image für Studenten


\subsection{Fernzugriff (VNC)}

OpenStack VNC Proxy

Screenshot

\subsection{Netzwerk}

\subsection{Installationsskript  für Benutzerkonfiguration}

Um die einzelnen Instanzen beim Erstellen auf einen Studenten anzupassen, muss das Image pro Student individuell verändert werden. 
Das \emph{cloud-init} Packet bietet die Möglichkeit mittels dem Parameter \emph{user-data} ein Skript beim Starten mitzugeben. 

Zur Installation der vorhergehend genannten Funktionen (LDAP-Authentifizierung etc.) wird ein generisches Skript generiert, welches nur noch minimal pro Student angepasst werden muss.
Dieses Skript \emph{setup-instance.sh} wird durch ein Makefile erstellt, in dem alle \emph{install.sh} Skripte der einzelnen Features aneinander gehängt und durch einen Header und Footer Teil ergänzt werden.
Im Header-Skript wird geprüft, ob root Rechte vorliegen und die \emph{USER} Variable gesetzt wurde und \emph{apt-get} aktualisiert. 
Der Footer ist dafür zuständig, nach dem Ausführen aller Teilskripte den Rechner neuzustarten.
Die \emph{USER} Variable muss vor dem Ausführen durch den tatsächlichen Rechenzentrum Benutzernamen des Studenten ersetzt werden.
Im Dashboard wird mittels Python das Skript angepasst und als user-data Parameter gesetzt.


% ------------------------------------------ ENDE Simon


\section{Probleme bei der Implementierung}

\section{Bekannte Bugs}


% ------------------------------------------ CHAPTER
\chapter{Deployment}

Kurzabriss des Kapitels

% ------------------------------------------ CHAPTER
\chapter{Ausblick und Fazit}

Kurzabriss des Kapitels

% ------------------------------------------ CHAPTER
\chapter{Kurzanleitung}

Kurzabriss des Kapitels

		
\bookmarksetup{startatroot}
%Literaturverzeichnis
	\pagenumbering{roman}
   	\cleardoublepage
   	\setcounter{page}{12}
	\bibliography{res/latexlit} 
	
% Anhang
\begin{appendix}   
\end{appendix}
\backmatter
\end{document}
